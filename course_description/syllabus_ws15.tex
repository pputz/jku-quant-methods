\documentclass[]{article}
\usepackage[T1]{fontenc}
\usepackage{lmodern}
\usepackage{amssymb,amsmath}
\usepackage{ifxetex,ifluatex}
\usepackage{fixltx2e} % provides \textsubscript
% use upquote if available, for straight quotes in verbatim environments
\IfFileExists{upquote.sty}{\usepackage{upquote}}{}
\ifnum 0\ifxetex 1\fi\ifluatex 1\fi=0 % if pdftex
  \usepackage[utf8]{inputenc}
\else % if luatex or xelatex
  \ifxetex
    \usepackage{mathspec}
    \usepackage{xltxtra,xunicode}
  \else
    \usepackage{fontspec}
  \fi
  \defaultfontfeatures{Mapping=tex-text,Scale=MatchLowercase}
  \newcommand{\euro}{€}
\fi
% use microtype if available
\IfFileExists{microtype.sty}{\usepackage{microtype}}{}
\usepackage{longtable,booktabs}
\ifxetex
  \usepackage[setpagesize=false, % page size defined by xetex
              unicode=false, % unicode breaks when used with xetex
              xetex]{hyperref}
\else
  \usepackage[unicode=true]{hyperref}
\fi
\hypersetup{breaklinks=true,
            bookmarks=true,
            pdfauthor={},
            pdftitle={},
            colorlinks=true,
            citecolor=blue,
            urlcolor=black,
            linkcolor=magenta,
            pdfborder={0 0 0}}
\urlstyle{same}  % don't use monospace font for urls
\setlength{\parindent}{0pt}
\setlength{\parskip}{6pt plus 2pt minus 1pt}
\setlength{\emergencystretch}{3em}  % prevent overfull lines
\setcounter{secnumdepth}{0}

\author{}
\date{}
\usepackage[a4paper, margin=3cm]{geometry}

\usepackage{tikz}
%\usepackage{tikzpagenodes}
\usetikzlibrary{positioning,calc}


\usepackage[naustrian,german,american]{babel}	% the last language is the default
\usepackage{csquotes}

% Chicago style
%\usepackage[authordate,backend=biber,babel=other]{biblatex-chicago}

%stadard biblatex styles
\usepackage[style=authoryear,sorting=ydnt,maxbibnames=99,backend=biber]{biblatex}
%\usepackage[style=authoryear,sorting=ydnt,maxbibnames=99,backend=biber,autolang=hyphen]{biblatex}

% apa style
%\usepackage[style=apa,backend=biber,babel=other]{biblatex}
%\DeclareLanguageMapping{naustrian}{naustrian-apa}
%\DeclareLanguageMapping{american}{american-apa}


\addbibresource{/Users/pp/Dropbox/Literature/academicDB.bib}
%\addbibresource{biblatex-examples.bib}

\begin{document}

\begin{tikzpicture}[remember picture,overlay]
\node [anchor=north west, outer sep=0]  at ($(current page.north)-(0.8cm, 0.6cm)$)
 {\includegraphics[width=8.3cm]{JKULogolangengl.eps}};
\end{tikzpicture}

\centering

\section{Master Seminar Quantitative Research
Methods}\label{master-seminar-quantitative-research-methods}

266.018, WS 2015\\\textbf{Dr.~Peter PUTZ}\\Institut für Organisation und
Globale Managementstudien\\peter.putz@jku.at\\Last changes: \today

\raggedright

\subsection{Seminar Description}\label{seminar-description}

This seminar will follow a modern data science approach. We will use R
-- one of the most popular statistical programming languages -- hands-on
throughout the course. You will learn how to apply statistical methods
to practical research questions in management. You will understand the
importance of reproducible research and create reproducible reports and
presentations.

\subsection{Entry Requirements}\label{entry-requirements}

\begin{itemize}
\item
  Completed courses: Master Course Dimensions of Marketing Theory and
  Managerial Application (Modul Marketing) and Master Course
  Organization (Modul Organization). This is a hard requirement.
\item
  A PC (MacOS or Windows) to install R and RStudio (both are available
  for free). We will use RStudio extensively throughout the course.
  Ideally, you would have a laptop to bring to our class meetings.
\item
  Programming skills are not required. However, you need to be prepared
  to learn a programming language and to write short computer scripts in
  R.
\item
  Basic understanding of quantitative research methods is highly
  recommended but not required. Students who do not have much prior
  knowledge in quantitative methods or in statistical programming will
  need to catch up by investing more time in (guided) self-study.
\end{itemize}

\subsection{Seminar Schedule and
Topics}\label{seminar-schedule-and-topics}

(see KUSSS for possible changes)

\begin{longtable}[c]{@{}rlll@{}}
\toprule\addlinespace
\begin{minipage}[b]{0.09\columnwidth}\raggedleft
Date
\end{minipage} & \begin{minipage}[b]{0.15\columnwidth}\raggedright
Time
\end{minipage} & \begin{minipage}[b]{0.13\columnwidth}\raggedright
Room
\end{minipage} & \begin{minipage}[b]{0.51\columnwidth}\raggedright
Topic
\end{minipage}
\\\addlinespace
\midrule\endhead
\begin{minipage}[t]{0.09\columnwidth}\raggedleft
8.10.
\end{minipage} & \begin{minipage}[t]{0.15\columnwidth}\raggedright
17:15--19:45
\end{minipage} & \begin{minipage}[t]{0.13\columnwidth}\raggedright
K 033C
\end{minipage} & \begin{minipage}[t]{0.51\columnwidth}\raggedright
Seminar overview, introduction to programming language R
\end{minipage}
\\\addlinespace
\begin{minipage}[t]{0.09\columnwidth}\raggedleft
15.10.
\end{minipage} & \begin{minipage}[t]{0.15\columnwidth}\raggedright
17:15--19:45
\end{minipage} & \begin{minipage}[t]{0.13\columnwidth}\raggedright
K 033C
\end{minipage} & \begin{minipage}[t]{0.51\columnwidth}\raggedright
Reproducible research, knitr
\end{minipage}
\\\addlinespace
\begin{minipage}[t]{0.09\columnwidth}\raggedleft
22.10.
\end{minipage} & \begin{minipage}[t]{0.15\columnwidth}\raggedright
17:15--19:45
\end{minipage} & \begin{minipage}[t]{0.13\columnwidth}\raggedright
K 033C
\end{minipage} & \begin{minipage}[t]{0.51\columnwidth}\raggedright
Exploratory data analysis, ggplot
\end{minipage}
\\\addlinespace
\begin{minipage}[t]{0.09\columnwidth}\raggedleft
29.10.
\end{minipage} & \begin{minipage}[t]{0.15\columnwidth}\raggedright
17:15--19:45
\end{minipage} & \begin{minipage}[t]{0.13\columnwidth}\raggedright
K 033C
\end{minipage} & \begin{minipage}[t]{0.51\columnwidth}\raggedright
Examples exploratory data analysis
\end{minipage}
\\\addlinespace
\begin{minipage}[t]{0.09\columnwidth}\raggedleft
5.11.
\end{minipage} & \begin{minipage}[t]{0.15\columnwidth}\raggedright
17:15--19:45
\end{minipage} & \begin{minipage}[t]{0.13\columnwidth}\raggedright
K 033C
\end{minipage} & \begin{minipage}[t]{0.51\columnwidth}\raggedright
Statistical inference
\end{minipage}
\\\addlinespace
\begin{minipage}[t]{0.09\columnwidth}\raggedleft
12.11.
\end{minipage} & \begin{minipage}[t]{0.15\columnwidth}\raggedright
17:15--19:45
\end{minipage} & \begin{minipage}[t]{0.13\columnwidth}\raggedright
K 033C
\end{minipage} & \begin{minipage}[t]{0.51\columnwidth}\raggedright
Regression models
\end{minipage}
\\\addlinespace
\begin{minipage}[t]{0.09\columnwidth}\raggedleft
19.11.
\end{minipage} & \begin{minipage}[t]{0.15\columnwidth}\raggedright
17:15--19:45
\end{minipage} & \begin{minipage}[t]{0.13\columnwidth}\raggedright
K 033C
\end{minipage} & \begin{minipage}[t]{0.51\columnwidth}\raggedright
Preparation for team papers, presentations
\end{minipage}
\\\addlinespace
\begin{minipage}[t]{0.09\columnwidth}\raggedleft
3.12.
\end{minipage} & \begin{minipage}[t]{0.15\columnwidth}\raggedright
17:15--19:45
\end{minipage} & \begin{minipage}[t]{0.13\columnwidth}\raggedright
K 033C
\end{minipage} & \begin{minipage}[t]{0.51\columnwidth}\raggedright
Team Presentations
\end{minipage}
\\\addlinespace
\bottomrule
\end{longtable}

\subsection{Grading}\label{grading}

\begin{itemize}
\itemsep1pt\parskip0pt\parsep0pt
\item
  Class Assignments: 60\%
\item
  Team research project (paper \& presentation): 40\%
\item
  A minimum of 80\% attendance is required.
\end{itemize}

\subsection{Literature}\label{literature}

\begin{enumerate}
\def\labelenumi{\arabic{enumi}.}
\item
  Textbooks on basics:

  \fullcite{diez2014}. A free PDF version is available at
  \href{https://www.openintro.org}{www.openintro.org}.

  \fullcite{zumel2014}.

  \fullcite{ghauri2010}. Some copies are available in the JKU library.

  Alternatively, for German speakers: \fullcite{lehner2012}.
\item
  Books on specific topics:

  \fullcite{peng2015d}

  \fullcite{leek2015}

  \fullcite{peng2015a}

  \fullcite{peng2015b}

  \fullcite{caffo2015a}

  \fullcite{caffo2015b}

  \fullcite{peng2015c}
\end{enumerate}

\end{document}
